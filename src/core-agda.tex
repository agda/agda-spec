\documentclass[acmlarge]{acmart}\settopmatter{}

%% Note: Authors migrating a paper from PACMPL format to traditional
%% SIGPLAN proceedings format should change 'acmlarge' to
%% 'sigplan,10pt'.


%% Some recommended packages.
\usepackage{booktabs}   %% For formal tables:
                        %% http://ctan.org/pkg/booktabs
\usepackage{subcaption} %% For complex figures with subfigures/subcaptions
                        %% http://ctan.org/pkg/subcaption
\usepackage{mathpartir}

\DeclareMathOperator{\dom}{dom}
\DeclareMathOperator{\proj}{proj}
\renewcommand{\|}{|}


\makeatletter\if@ACM@journal\makeatother
%% Journal information (used by PACMPL format)
%% Supplied to authors by publisher for camera-ready submission
\acmJournal{PACMPL}
\acmVolume{1}
\acmNumber{1}
\acmArticle{1}
\acmYear{2017}
\acmMonth{1}
\acmDOI{10.1145/nnnnnnn.nnnnnnn}
\startPage{1}
\else\makeatother
%% Conference information (used by SIGPLAN proceedings format)
%% Supplied to authors by publisher for camera-ready submission
\acmConference[PL'18]{ACM SIGPLAN Conference on Programming Languages}{January 01--03, 2018}{New York, NY, USA}
\acmYear{2018}
\acmISBN{978-x-xxxx-xxxx-x/YY/MM}
\acmDOI{10.1145/nnnnnnn.nnnnnnn}
\startPage{1}
\fi


%% Copyright information
%% Supplied to authors (based on authors' rights management selection;
%% see authors.acm.org) by publisher for camera-ready submission
\setcopyright{none}             %% For review submission
%\setcopyright{acmcopyright}
%\setcopyright{acmlicensed}
%\setcopyright{rightsretained}
%\copyrightyear{2017}           %% If different from \acmYear


%% Bibliography style
\bibliographystyle{ACM-Reference-Format}
%% Citation style
%% Note: author/year citations are required for papers published as an
%% issue of PACMPL.
\citestyle{acmauthoryear}   %% For author/year citations

\RequirePackage{ifthen}
\RequirePackage{amssymb}
\RequirePackage{amsfonts}
\RequirePackage{stmaryrd} % \shortuparrow


% evergreens
\newcommand{\bla}{\ensuremath{\mbox{$$}}} % invisible, but not ignored
\newcommand{\der}{\,\vdash}
\newcommand{\of}{\!:\!}
\newcommand{\is}{\!=\!}
\newcommand{\red}{\longrightarrow}
\newcommand{\restrict}{\upharpoonright}
\newcommand{\FV}{\ensuremath{\mathsf{FV}}}
\newcommand{\NN}{\mathbb{N}}
\newcommand{\defas}{\mathrel{\ :\Longleftrightarrow\ }}
\newcommand{\defiff}{\mathrel{:\Longleftrightarrow}}
\DeclareMathOperator{\dom}{dom}

% latin etc. abbrev
\newcommand{\abbrev}[1]{#1} % alternative: \emph{#1}
\newcommand{\cf}{\abbrev{cf.}\ }
\newcommand{\eg}{\abbrev{e.\,g.}}
\newcommand{\Eg}{\abbrev{E.\,g.}}
\newcommand{\ie}{\abbrev{i.\,e.}}
\newcommand{\Ie}{\abbrev{I.\,e.}}
\newcommand{\etal}{\abbrev{et.\,al.}}
\newcommand{\wwlog}{w.\,l.\,o.\,g.} % \wlog is ``write into log file''
\newcommand{\Wlog}{W.\,l.\,o.\,g.}
\newcommand{\wrt}{w.\,r.\,t.}

% paragraphs
\newcommand{\para}[1]{\paragraph*{\it#1}}
\newcommand{\paradot}[1]{\para{#1.}}

% proof by cases
\newenvironment{caselist}{%
  \begin{list}{{\it Case}}{%
    %\setlength{\topsep}{2ex}% DOES NOT SEEM TO WORK
    %\setlength{\itemsep}{2ex}%
    \setlength{\itemindent}{-2ex}%
  }%
}{\end{list}%
}
\newenvironment{subcaselist}{%
  \begin{list}{{\it Subcase}}{}%
}{\end{list}%
}
\newenvironment{subsubcaselist}{%
  \begin{list}{{\it Subsubcase}}{}%
}{\end{list}%
}

\newcommand{\nextcase}{\item~}

% meta-level logic
\newcommand{\mfor}{\ \mbox{for}\ }
\newcommand{\mforsome}{\ \mbox{for some}\ }
\newcommand{\mthen}{\ \mbox{then}\ }
\newcommand{\mif}{\ \mbox{if}\ }
\newcommand{\miff}{\ \mbox{iff}\ }
\newcommand{\motherwise}{\ \mbox{otherwise}}
\newcommand{\mundefined}{\mbox{undefined}}
\newcommand{\mnot}{\mbox{not}\ }
\newcommand{\mand}{\ \mbox{and}\ }
\newcommand{\mor}{\ \mbox{or}\ }
\newcommand{\mimplies}{\ \mbox{implies}\ }
\newcommand{\mimply}{\ \mbox{imply}\ }
\newcommand{\mforall}{\ \mbox{for all}\ }
\newcommand{\mexists}{\mbox{exists}\ }
\newcommand{\mexist}{\mbox{exist}\ }
\newcommand{\mtrue}{\mbox{true}}
\newcommand{\mwith}{\ \mbox{with}\ }
\newcommand{\mholds}{\ \mbox{holds}\ }
\newcommand{\munless}{\ \mbox{unless}\ }
\newcommand{\mboth}{\ \mbox{both}\ }
\newcommand{\msuchthat}{\ \mbox{such that}\ }
% proofs
\newcommand{\msince}{\mbox{since}\ }
\newcommand{\mdef}{\mbox{by def.}}
\newcommand{\mass}{\mbox{assumption}}
\newcommand{\mhyp}{\mbox{by hyp.}}
\newcommand{\mlemma}[1]{\mbox{by Lemma~#1}}
\newcommand{\mih}[1][]{\mbox{by ind.hyp.}#1}
\newcommand{\mgoal}[1][]{\mbox{goal\ifthenelse{\equal{#1}{}}{}{~#1}}}
\newcommand{\mby}[1]{\mbox{by #1}}
\newcommand{\minfrule}{\mbox{by inference rule}}


% Inference rules
\newcommand{\rulename}[1]{\ensuremath{\mbox{\sc#1}}}
\newcommand{\rul}[2]{\dfrac{\begin{array}[b]{@{}l@{}} #1 \end{array}}{#2}}
\newcommand{\ru}[2]{\dfrac{\begin{array}[b]{@{}c@{}} #1 \end{array}}{\begin{array}[l]{@{}c@{}} #2 \end{array}}}
\newcommand{\rux}[3]{\ru{#1}{#2}\ #3}
\newcommand{\nru}[3]{#1\ \ru{#2}{#3}}
\newcommand{\nrux}[4]{#1\ \ru{#2}{#3}\ #4}
\newcommand{\dstack}[2]{\begin{array}[b]{c}#1\\#2\end{array}}
\newcommand{\ndru}[4]{#1\ \ru{\dstack{#2}{#3}}{#4}}
\newcommand{\ndrux}[5]{#1\ \ru{\dstack{#2}{#3}}{#4}\ #5}
\newcommand{\lcol}[1]{\multicolumn{1}{@{}l@{}}{{#1}}}
\newcommand{\rcol}[1]{\multicolumn{1}{@{}r@{}}{{#1}}}

% Substitution and function update
% read ``\subst F X A'' as ``substitute F for X in A''
%\newcommand{\subst}[3]{#3[#2 := #1]}
%\newcommand{\subst}[3]{[#1/#2]#3}
\newcommand{\subst}[3]{#3[#1/#2]}
% read ``\update \theta X \G'' as update \theta at point X by \G
\newcommand{\update}[3]{#1[#2 \mapsto #3]}
%\newcommand{\update}[3]{#1,#2 \is #3}


% Core agda syntax
\newcommand{\funT}[3]{(#1 : #2) \to #3}
\newcommand{\piT}{\funT}
%\newcommand{\piT}[3]{\Pi #1 \of #2.\, #3}
\newcommand{\lam}[1]{\lambda #1.\,}
\newcommand{\Set}{\mathsf{Set}}
\newcommand{\cpi}{c_{\vec\pi}}  % record constructor
\newcommand{\cempty}{\ensuremath{\mathord{\cdot}}}
\newcommand{\dotp}[1]{\lfloor#1\rfloor} % inaccessible pattern
\newcommand{\embp}[1]{\lceil#1\rceil} % embedding of patterns into  terms

\newcommand{\datasig}{\text{data } D\; \Delta : \Delta' \to s \text{ where } \overline{c : T}}
\newcommand{\recsig}{\text{record } R\;\Delta : s \text{ where
    $c : T$; } \text{projections } \overline{\pi : T} }
\newcommand{\funsig}{\text{function } f : T \text{ where } \overline{cl}}

\DeclareMathOperator{\proj}{proj}

% Variable names
\newcommand{\vts}{\mathit{ts}}
\newcommand{\vcl}{\mathit{cl}}
\newcommand{\vds}{\mathit{ds}}
\newcommand{\vdd}{\mathit{dd}}
\newcommand{\vrs}{\mathit{rs}}
\newcommand{\vrd}{\mathit{rd}}

% Core agda judgements
\newcommand{\ders}{\der_\Sigma}
\newcommand{\bang}{\mathrel{!}}
\newcommand{\twobang}{\mathrel{!!}}

%%% Local Variables:
%%% mode: latex
%%% TeX-master: "core-agda"
%%% End:



\begin{document}

%% Title information
\title[Core Agda]{Specification of Core Agda}         %% [Short Title] is optional;
                                        %% when present, will be used in
                                        %% header instead of Full Title.
\titlenote{with title note}             %% \titlenote is optional;
                                        %% can be repeated if necessary;
                                        %% contents suppressed with 'anonymous'
\subtitle{Subtitle}                     %% \subtitle is optional
\subtitlenote{with subtitle note}       %% \subtitlenote is optional;
                                        %% can be repeated if necessary;
                                        %% contents suppressed with 'anonymous'
 x

%% Author information
%% Contents and number of authors suppressed with 'anonymous'.
%% Each author should be introduced by \author, followed by
%% \authornote (optional), \orcid (optional), \affiliation, and
%% \email.
%% An author may have multiple affiliations and/or emails; repeat the
%% appropriate command.
%% Many elements are not rendered, but should be provided for metadata
%% extraction tools.

%% Author with single affiliation.
\author{First1 Last1}
\authornote{with author1 note}          %% \authornote is optional;
                                        %% can be repeated if necessary
\orcid{nnnn-nnnn-nnnn-nnnn}             %% \orcid is optional
\affiliation{
  \position{Position1}
  \department{Department1}              %% \department is recommended
  \institution{Institution1}            %% \institution is required
  \streetaddress{Street1 Address1}
  \city{City1}
  \state{State1}
  \postcode{Post-Code1}
  \country{Country1}
}
\email{first1.last1@inst1.edu}          %% \email is recommended

%% Author with two affiliations and emails.
\author{First2 Last2}
\authornote{with author2 note}          %% \authornote is optional;
                                        %% can be repeated if necessary
\orcid{nnnn-nnnn-nnnn-nnnn}             %% \orcid is optional
\affiliation{
  \position{Position2a}
  \department{Department2a}             %% \department is recommended
  \institution{Institution2a}           %% \institution is required
  \streetaddress{Street2a Address2a}
  \city{City2a}
  \state{State2a}
  \postcode{Post-Code2a}
  \country{Country2a}
}
\email{first2.last2@inst2a.com}         %% \email is recommended
\affiliation{
  \position{Position2b}
  \department{Department2b}             %% \department is recommended
  \institution{Institution2b}           %% \institution is required
  \streetaddress{Street3b Address2b}
  \city{City2b}
  \state{State2b}
  \postcode{Post-Code2b}
  \country{Country2b}
}
\email{first2.last2@inst2b.org}         %% \email is recommended


%% Paper note
%% The \thanks command may be used to create a "paper note" ---
%% similar to a title note or an author note, but not explicitly
%% associated with a particular element.  It will appear immediately
%% above the permission/copyright statement.
\thanks{with paper note}                %% \thanks is optional
                                        %% can be repeated if necesary
                                        %% contents suppressed with 'anonymous'


%% Abstract
%% Note: \begin{abstract}...\end{abstract} environment must come
%% before \maketitle command
\begin{abstract}
This document specifies the abstract syntax, evaluation rules, and
typing rules of Core Agda, the basic type theory underlying Agda.
\end{abstract}


%% 2012 ACM Computing Classification System (CSS) concepts
%% Generate at 'http://dl.acm.org/ccs/ccs.cfm'.
\begin{CCSXML}
<ccs2012>
<concept>
<concept_id>10011007.10011006.10011008</concept_id>
<concept_desc>Software and its engineering~General programming languages</concept_desc>
<concept_significance>500</concept_significance>
</concept>
<concept>
<concept_id>10003456.10003457.10003521.10003525</concept_id>
<concept_desc>Social and professional topics~History of programming languages</concept_desc>
<concept_significance>300</concept_significance>
</concept>
</ccs2012>
\end{CCSXML}

\ccsdesc[500]{Software and its engineering~General programming languages}
\ccsdesc[300]{Social and professional topics~History of programming languages}
%% End of generated code


%% Keywords
%% comma separated list
\keywords{Agda, dependent types, specification}  %% \keywords is optional


%% \maketitle
%% Note: \maketitle command must come after title commands, author
%% commands, abstract environment, Computing Classification System
%% environment and commands, and keywords command.
\maketitle


\section{Introduction}

Agda 2 has been developed since 2005, and been released 2007.  So far,
no specification has been given.  This document attempts to specify
the core components of Agda.


\section{Syntax}
\label{sec:syntax}

\newcommand{\bang}{\,!\,}
\newcommand{\twobang}{\,!!\,}


Names
\[
\begin{array}{ll}
D   & \hspace{2cm}\text{Datatype name} \\
  R & \hspace{2cm}\text{   Record type name} \\
  f & \hspace{2cm}\text{   Function name} \\
\pi & \hspace{2cm}\text{  Projection name  (overloaded)} \\
  c & \hspace{2cm}\text{   Constructor name (overloaded)} \\
\end{array}
\]

Variables (represented by a de Bruijn-index)
\begin{verbatim}
x
\end{verbatim}

Data/record type name
\begin{verbatim}
F ::= D | R
\end{verbatim}

Terms
\begin{verbatim}
t, u, v ::= x e*            Variables (eliminated by e*)
          | c v*            Constructor applied to arguments v*
          | f e*            Defined function (eliminated by e*), includes postulates
          | \x. v           Lambda abstraction
          | F v*            Data/record type
          | s               Sort
          | (x : A) -> B    Dependent function type
\end{verbatim}

Eliminations
\begin{verbatim}
e ::= @u                Application to term
    | .π                Projection
\end{verbatim}

Sorts
\begin{verbatim}
s ::= Set_n             Universe of types of level n
\end{verbatim}

Telescopes
\begin{verbatim}
Γ , Δ ::= · | (x : T)Δ
\end{verbatim}

Iterated function types Γ → T
\begin{verbatim}
· → T = T
(x : U)Δ → T = (x : U) → (Δ → T)
\end{verbatim}

\section{Declarations}
\label{sec:declarations}

Declarations
\begin{verbatim}

\end{verbatim}

Type signatures
\begin{verbatim}
ts ::= f : T
\end{verbatim}

Function clauses
\begin{verbatim}
cl ::= f q* = t
\end{verbatim}

Copattern
\begin{verbatim}
q ::= @p | .π 
\end{verbatim}

Pattern
\begin{verbatim}
p ::= x       Variable pattern
    | c p*    Constructor pattern
    | .u      Dot pattern
\end{verbatim}

Data definition
\begin{verbatim}
data D Γ : Δ → s where 
  (c : T)*
\end{verbatim}

Record declaration
\begin{verbatim}
record R Γ : s where
  constructor c
  field (π : T)*
\end{verbatim}

Mutual block
\begin{verbatim}
mutual d*
\end{verbatim}

Signature declaration
\[
\begin{array}{lcl}
d_\Sigma & ::= & \text{data } D\Gamma: \Delta \to S \text{ where } \overline{c : T} \\
& \| & \text{record } R\Gamma : S \text{ where constructor $c$; } \overline{\text{field }\pi : T} \\
& \| & \text{function } f : T \text{ where } \overline{cl}
\end{array}
\]


Signature
\[
\begin{array}{lcl}
\Sigma & ::= &  \overline{d_\Sigma} \\
\end{array}
\]


\section{Evaluation}
\label{sec:evaluation}

Hereditary substitution
\[
\begin{array}{lll}
u[\sigma] = v & t \bang \overline{e} = v & t \bang \cdot = t \\
D \overline{v} \bang @ \overline{u} = D \overline{v} \overline{u}
& R \overline{v} \bang @ \overline{u} = R \overline{v} \overline{u}
\end{array}
\]

\[
\begin{array}{ll}
\inferrule{\sigma(x) \bang \overline{e}[\sigma] = v}
{x \overline{e}[\sigma] = v}
&
\inferrule{\overline{e}[\sigma] = \overline{e}'}
{f \overline{e}[\sigma] = f \overline{e}'}
\\
\inferrule{ v[u/x] \bang \overline{e} = v'}
{\lambda x.v \bang @u \overline{e} = v'}
&
\inferrule{v_i ! \overline{e} = v'}
{c_{\overline{pi}} \overline{v} \bang .\pi_i \overline{e} = v'}
\\
\inferrule{c \overline{v} u \bang \overline{e} = v'}
{\overline{v} \bang @u \overline{e} = v'}
&
\inferrule{}
{f \overline{e} \bang \overline{e}' = f \overline{e} \overline{e}'}
\\
\inferrule{}
{x \overline{e} \bang \overline{e}' = x \overline{e} \overline{e}'}
&
\text{ everything else undefined }
\end{array}
\]



Matching
\[
\begin{array}{ll}
\overline{e}/\overline{q} = \sigma  & 
\Sigma \vdash e/q = \sigma
\end{array}
\]

Evaluation
\[
\begin{array}{ll}
\inferrule{}
{\Sigma \vdash v/\lfloor u \rfloor = \cdot} % dot pattern
& 
\inferrule{}
{\Sigma \vdash v/x = [v/x]} % var pat
\\
\inferrule{\Sigma \vdash \overline{v}/\overline{\sigma} = \sigma}
{\Sigma \vdash c\overline{v}/c\overline{p} = \sigma}  % con pat
&
\inferrule{}
{\Sigma \vdash .\pi/.\pi = \cdot} % proj pat
\\
\inferrule{\Sigma \vdash v \to v' \qquad \Sigma \vdash v'/c\overline{p} = \sigma} 
{\Sigma \vdash v / c \overline{p} = \sigma} % reduce
& 
\inferrule{v \bang \pi_i / p_i = \sigma}
{\Sigma \vdash v / c_{\overline \pi} \overline{p} = \sigma} % record pat
\end{array}
\]

Weak head reduction
\[
\begin{array}{ll}
\Sigma \vdash t \to t' &
\inferrule{f \overline q = t \in \Sigma \qquad \Sigma \vdash \overline{e}/\overline{q} = \sigma}
{\Sigma \vdash f \overline{e} \overline{e}' \to t[\sigma] \bang \overline{e}'}
\\
\inferrule{\Sigma \vdash e/q = \sigma \qquad \Sigma \vdash \overline{e}/\overline{q} = \sigma'}
{\Sigma \vdash e, \overline{e}/q, \overline{q} = \sigma \overset{\cup}{+} \sigma'}%patterns
&
\inferrule{}
{\Sigma \vdash \cdot/\cdot = \cdot} %empty
\end{array}
\]


\section{Typing rules}
\label{sec:typing}

Context 
\[
\begin{array}{lcl}
\Gamma & ::= & \cdot \| \Gamma, x:T \\
\Delta & ::= & \cdot \| x:T, \Delta
\end{array}
\]

Typing 
\[
\begin{array}{ll}
\Gamma \vdash_\Sigma t : T & \Gamma \| u: U \vdash \overline{e} : T
\end{array}
\]

Conversion 
\[
\begin{array}{ll}
\Gamma \vdash_\Sigma t = t' : T & \Gamma \vdash_\Sigma \Delta \\
\vdash_\Sigma \Gamma & \\

% Contexts
\inferrule{}
{\vdash_\Sigma \cdot} & 
\inferrule{\vdash_\Sigma \Gamma \qquad \Gamma \vdash_\Sigma T:s}
{\vdash_\Sigma \Gamma, x : T} \qquad x \not\in \dom(\Gamma) \\

% Telescopes
\inferrule{\vdash_\Sigma \Gamma}
{\vdash_\Sigma \cdot} &
\inferrule{\Gamma \vdash_\Sigma T : s \qquad \Gamma, x:T \vdash_\Sigma \Delta}
{\Gamma \vdash_\Sigma (x:T) \Delta}
\qquad x \in \dom{\Delta} \\

%Terms 
\inferrule{(x:U)\in \Gamma \qquad \Gamma | x:U \vdash \overline{e}:T}
{\Gamma \vdash x \overline{e} : T} 
&
% Check spine
\inferrule{}
{\Gamma \| u:U \vdash \cdot : U}
\\
\inferrule{\Gamma \vdash u:U \qquad \Gamma \| t@u:T[u/x] \vdash \overline{e : V}}
{\Gamma \| t : \Pi x:U.T \vdash @u \overline{e}:V}
&
\inferrule{ \Gamma \vdash (t:R\overline{u}).\pi = T' 
\qquad \Gamma \| t.\pi :T' \vdash_\Sigma \overline{e}:V}
{ \Gamma \| t : R \overline{u}  \vdash_\Sigma .\pi \overline{e}:V}
\\
\Sigma \vdash (t:R\overline{u}).\pi = T
\\
(t: \Sigma x:A.B).\proj_2 = B[t.\proj_1/x] &
\\
\inferrule{T[u/x] \twobang \overline{e} = v}
{(\Pi x:U.T) \twobang @ u \overline{e} = v}
\end{array}
\]


\section{Coverage}
\label{sec:coverage}

\section{Termination}
\label{sec:termination}

\section{Positivity}
\label{sec:positivity}

\section{Extensions}
\label{sec:extensions}

\subsection{Extended record declarations}

Record types
\begin{verbatim}
record R Γ : s where
  [constructor c]
  [(co)inductive]
  [(no-)eta-equality]
  rd*
\end{verbatim}

Record declaration
\begin{verbatim}
rd ::= field π : T
     | ts
     | cl 
\end{verbatim}


%% Acknowledgments
\begin{acks}                            %% acks environment is optional
                                        %% contents suppressed with 'anonymous'
  %% Commands \grantsponsor{<sponsorID>}{<name>}{<url>} and
  %% \grantnum[<url>]{<sponsorID>}{<number>} should be used to
  %% acknowledge financial support and will be used by metadata
  %% extraction tools.
  This material is based upon work supported by the
  \grantsponsor{GS100000001}{National Science
    Foundation}{http://dx.doi.org/10.13039/100000001} under Grant
  No.~\grantnum{GS100000001}{nnnnnnn} and Grant
  No.~\grantnum{GS100000001}{mmmmmmm}.  Any opinions, findings, and
  conclusions or recommendations expressed in this material are those
  of the author and do not necessarily reflect the views of the
  National Science Foundation.
\end{acks}


%% Bibliography
%\bibliography{bibfile}


%% Appendix
% \appendix
% \section{Appendix}


\end{document}
