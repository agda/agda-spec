\documentclass[acmlarge,fleqn]{acmart}\settopmatter{}

%% Note: Authors migrating a paper from PACMPL format to traditional
%% SIGPLAN proceedings format should change 'acmlarge' to
%% 'sigplan,10pt'.


%% Some recommended packages.
\usepackage{booktabs}   %% For formal tables:
                        %% http://ctan.org/pkg/booktabs
\usepackage{subcaption} %% For complex figures with subfigures/subcaptions
                        %% http://ctan.org/pkg/subcaption
\usepackage{mathpartir}


\makeatletter\if@ACM@journal\makeatother
%% Journal information (used by PACMPL format)
%% Supplied to authors by publisher for camera-ready submission
\acmJournal{PACMPL}
\acmVolume{1}
\acmNumber{1}
\acmArticle{1}
\acmYear{2017}
\acmMonth{1}
\acmDOI{10.1145/nnnnnnn.nnnnnnn}
\startPage{1}
\else\makeatother
%% Conference information (used by SIGPLAN proceedings format)
%% Supplied to authors by publisher for camera-ready submission
\acmConference[PL'18]{ACM SIGPLAN Conference on Programming Languages}{January 01--03, 2018}{New York, NY, USA}
\acmYear{2018}
\acmISBN{978-x-xxxx-xxxx-x/YY/MM}
\acmDOI{10.1145/nnnnnnn.nnnnnnn}
\startPage{1}
\fi


%% Copyright information
%% Supplied to authors (based on authors' rights management selection;
%% see authors.acm.org) by publisher for camera-ready submission
\setcopyright{none}             %% For review submission
%\setcopyright{acmcopyright}
%\setcopyright{acmlicensed}
%\setcopyright{rightsretained}
%\copyrightyear{2017}           %% If different from \acmYear


%% Bibliography style
\bibliographystyle{ACM-Reference-Format}
%% Citation style
%% Note: author/year citations are required for papers published as an
%% issue of PACMPL.
\citestyle{acmauthoryear}   %% For author/year citations

\RequirePackage{ifthen}
\RequirePackage{amssymb}
\RequirePackage{amsfonts}
\RequirePackage{stmaryrd} % \shortuparrow


% evergreens
\newcommand{\bla}{\ensuremath{\mbox{$$}}} % invisible, but not ignored
\newcommand{\der}{\,\vdash}
\newcommand{\of}{\!:\!}
\newcommand{\is}{\!=\!}
\newcommand{\red}{\longrightarrow}
\newcommand{\restrict}{\upharpoonright}
\newcommand{\FV}{\ensuremath{\mathsf{FV}}}
\newcommand{\NN}{\mathbb{N}}
\newcommand{\defas}{\mathrel{\ :\Longleftrightarrow\ }}
\newcommand{\defiff}{\mathrel{:\Longleftrightarrow}}
\DeclareMathOperator{\dom}{dom}

% latin etc. abbrev
\newcommand{\abbrev}[1]{#1} % alternative: \emph{#1}
\newcommand{\cf}{\abbrev{cf.}\ }
\newcommand{\eg}{\abbrev{e.\,g.}}
\newcommand{\Eg}{\abbrev{E.\,g.}}
\newcommand{\ie}{\abbrev{i.\,e.}}
\newcommand{\Ie}{\abbrev{I.\,e.}}
\newcommand{\etal}{\abbrev{et.\,al.}}
\newcommand{\wwlog}{w.\,l.\,o.\,g.} % \wlog is ``write into log file''
\newcommand{\Wlog}{W.\,l.\,o.\,g.}
\newcommand{\wrt}{w.\,r.\,t.}

% paragraphs
\newcommand{\para}[1]{\paragraph*{\it#1}}
\newcommand{\paradot}[1]{\para{#1.}}

% proof by cases
\newenvironment{caselist}{%
  \begin{list}{{\it Case}}{%
    %\setlength{\topsep}{2ex}% DOES NOT SEEM TO WORK
    %\setlength{\itemsep}{2ex}%
    \setlength{\itemindent}{-2ex}%
  }%
}{\end{list}%
}
\newenvironment{subcaselist}{%
  \begin{list}{{\it Subcase}}{}%
}{\end{list}%
}
\newenvironment{subsubcaselist}{%
  \begin{list}{{\it Subsubcase}}{}%
}{\end{list}%
}

\newcommand{\nextcase}{\item~}

% meta-level logic
\newcommand{\mfor}{\ \mbox{for}\ }
\newcommand{\mforsome}{\ \mbox{for some}\ }
\newcommand{\mthen}{\ \mbox{then}\ }
\newcommand{\mif}{\ \mbox{if}\ }
\newcommand{\miff}{\ \mbox{iff}\ }
\newcommand{\motherwise}{\ \mbox{otherwise}}
\newcommand{\mundefined}{\mbox{undefined}}
\newcommand{\mnot}{\mbox{not}\ }
\newcommand{\mand}{\ \mbox{and}\ }
\newcommand{\mor}{\ \mbox{or}\ }
\newcommand{\mimplies}{\ \mbox{implies}\ }
\newcommand{\mimply}{\ \mbox{imply}\ }
\newcommand{\mforall}{\ \mbox{for all}\ }
\newcommand{\mexists}{\mbox{exists}\ }
\newcommand{\mexist}{\mbox{exist}\ }
\newcommand{\mtrue}{\mbox{true}}
\newcommand{\mwith}{\ \mbox{with}\ }
\newcommand{\mholds}{\ \mbox{holds}\ }
\newcommand{\munless}{\ \mbox{unless}\ }
\newcommand{\mboth}{\ \mbox{both}\ }
\newcommand{\msuchthat}{\ \mbox{such that}\ }
% proofs
\newcommand{\msince}{\mbox{since}\ }
\newcommand{\mdef}{\mbox{by def.}}
\newcommand{\mass}{\mbox{assumption}}
\newcommand{\mhyp}{\mbox{by hyp.}}
\newcommand{\mlemma}[1]{\mbox{by Lemma~#1}}
\newcommand{\mih}[1][]{\mbox{by ind.hyp.}#1}
\newcommand{\mgoal}[1][]{\mbox{goal\ifthenelse{\equal{#1}{}}{}{~#1}}}
\newcommand{\mby}[1]{\mbox{by #1}}
\newcommand{\minfrule}{\mbox{by inference rule}}


% Inference rules
\newcommand{\rulename}[1]{\ensuremath{\mbox{\sc#1}}}
\newcommand{\rul}[2]{\dfrac{\begin{array}[b]{@{}l@{}} #1 \end{array}}{#2}}
\newcommand{\ru}[2]{\dfrac{\begin{array}[b]{@{}c@{}} #1 \end{array}}{\begin{array}[l]{@{}c@{}} #2 \end{array}}}
\newcommand{\rux}[3]{\ru{#1}{#2}\ #3}
\newcommand{\nru}[3]{#1\ \ru{#2}{#3}}
\newcommand{\nrux}[4]{#1\ \ru{#2}{#3}\ #4}
\newcommand{\dstack}[2]{\begin{array}[b]{c}#1\\#2\end{array}}
\newcommand{\ndru}[4]{#1\ \ru{\dstack{#2}{#3}}{#4}}
\newcommand{\ndrux}[5]{#1\ \ru{\dstack{#2}{#3}}{#4}\ #5}
\newcommand{\lcol}[1]{\multicolumn{1}{@{}l@{}}{{#1}}}
\newcommand{\rcol}[1]{\multicolumn{1}{@{}r@{}}{{#1}}}

% Substitution and function update
% read ``\subst F X A'' as ``substitute F for X in A''
%\newcommand{\subst}[3]{#3[#2 := #1]}
%\newcommand{\subst}[3]{[#1/#2]#3}
\newcommand{\subst}[3]{#3[#1/#2]}
% read ``\update \theta X \G'' as update \theta at point X by \G
\newcommand{\update}[3]{#1[#2 \mapsto #3]}
%\newcommand{\update}[3]{#1,#2 \is #3}


% Core agda syntax
\newcommand{\funT}[3]{(#1 : #2) \to #3}
\newcommand{\piT}{\funT}
%\newcommand{\piT}[3]{\Pi #1 \of #2.\, #3}
\newcommand{\lam}[1]{\lambda #1.\,}
\newcommand{\Set}{\mathsf{Set}}
\newcommand{\cpi}{c_{\vec\pi}}  % record constructor
\newcommand{\cempty}{\ensuremath{\mathord{\cdot}}}
\newcommand{\dotp}[1]{\lfloor#1\rfloor} % inaccessible pattern
\newcommand{\embp}[1]{\lceil#1\rceil} % embedding of patterns into  terms

\newcommand{\datasig}{\text{data } D\; \Delta : \Delta' \to s \text{ where } \overline{c : T}}
\newcommand{\recsig}{\text{record } R\;\Delta : s \text{ where
    $c : T$; } \text{projections } \overline{\pi : T} }
\newcommand{\funsig}{\text{function } f : T \text{ where } \overline{cl}}

\DeclareMathOperator{\proj}{proj}

% Variable names
\newcommand{\vts}{\mathit{ts}}
\newcommand{\vcl}{\mathit{cl}}
\newcommand{\vds}{\mathit{ds}}
\newcommand{\vdd}{\mathit{dd}}
\newcommand{\vrs}{\mathit{rs}}
\newcommand{\vrd}{\mathit{rd}}

% Core agda judgements
\newcommand{\ders}{\der_\Sigma}
\newcommand{\bang}{\mathrel{!}}
\newcommand{\twobang}{\mathrel{!!}}

%%% Local Variables:
%%% mode: latex
%%% TeX-master: "core-agda"
%%% End:


% COVERED by option fleqn:
%
% controlling flushleft/center for math displays
% http://www.golatex.de/wechsel-zwischen-leqno-und-reqno-fleqn-uvm-t2488.html
\makeatletter
%\def\leqn{\tagsleft@true}
%\def\reqn{\tagsleft@false}
\def\fleq{\@fleqntrue\let\mathindent\@mathmargin \@mathmargin=\leftmargini}
\def\cneq{\@fleqnfalse}
\g@addto@macro{\endsubequations}{\addtocounter{equation}{-1}}
\makeatother

\begin{document}

%% Title information
\title[Core Agda]{Specification of Core Agda}         %% [Short Title] is optional;
                                        %% when present, will be used in
                                        %% header instead of Full Title.
\titlenote{with title note}             %% \titlenote is optional;
                                        %% can be repeated if necessary;
                                        %% contents suppressed with 'anonymous'
\subtitle{Subtitle}                     %% \subtitle is optional
\subtitlenote{with subtitle note}       %% \subtitlenote is optional;
                                        %% can be repeated if necessary;
                                        %% contents suppressed with 'anonymous'


%% Author information
%% Contents and number of authors suppressed with 'anonymous'.
%% Each author should be introduced by \author, followed by
%% \authornote (optional), \orcid (optional), \affiliation, and
%% \email.
%% An author may have multiple affiliations and/or emails; repeat the
%% appropriate command.
%% Many elements are not rendered, but should be provided for metadata
%% extraction tools.

%% Author with single affiliation.
\author{First1 Last1}
\authornote{with author1 note}          %% \authornote is optional;
                                        %% can be repeated if necessary
\orcid{nnnn-nnnn-nnnn-nnnn}             %% \orcid is optional
\affiliation{
  \position{Position1}
  \department{Department1}              %% \department is recommended
  \institution{Institution1}            %% \institution is required
  \streetaddress{Street1 Address1}
  \city{City1}
  \state{State1}
  \postcode{Post-Code1}
  \country{Country1}
}
\email{first1.last1@inst1.edu}          %% \email is recommended

%% Author with two affiliations and emails.
\author{First2 Last2}
\authornote{with author2 note}          %% \authornote is optional;
                                        %% can be repeated if necessary
\orcid{nnnn-nnnn-nnnn-nnnn}             %% \orcid is optional
\affiliation{
  \position{Position2a}
  \department{Department2a}             %% \department is recommended
  \institution{Institution2a}           %% \institution is required
  \streetaddress{Street2a Address2a}
  \city{City2a}
  \state{State2a}
  \postcode{Post-Code2a}
  \country{Country2a}
}
\email{first2.last2@inst2a.com}         %% \email is recommended
\affiliation{
  \position{Position2b}
  \department{Department2b}             %% \department is recommended
  \institution{Institution2b}           %% \institution is required
  \streetaddress{Street3b Address2b}
  \city{City2b}
  \state{State2b}
  \postcode{Post-Code2b}
  \country{Country2b}
}
\email{first2.last2@inst2b.org}         %% \email is recommended


%% Paper note
%% The \thanks command may be used to create a "paper note" ---
%% similar to a title note or an author note, but not explicitly
%% associated with a particular element.  It will appear immediately
%% above the permission/copyright statement.
\thanks{with paper note}                %% \thanks is optional
                                        %% can be repeated if necesary
                                        %% contents suppressed with 'anonymous'


%% Abstract
%% Note: \begin{abstract}...\end{abstract} environment must come
%% before \maketitle command
\begin{abstract}
This document specifies the abstract syntax, evaluation rules, and
typing rules of Core Agda, the basic type theory underlying Agda.
\end{abstract}


%% 2012 ACM Computing Classification System (CSS) concepts
%% Generate at 'http://dl.acm.org/ccs/ccs.cfm'.
\begin{CCSXML}
<ccs2012>
<concept>
<concept_id>10011007.10011006.10011008</concept_id>
<concept_desc>Software and its engineering~General programming languages</concept_desc>
<concept_significance>500</concept_significance>
</concept>
<concept>
<concept_id>10003456.10003457.10003521.10003525</concept_id>
<concept_desc>Social and professional topics~History of programming languages</concept_desc>
<concept_significance>300</concept_significance>
</concept>
</ccs2012>
\end{CCSXML}

\ccsdesc[500]{Software and its engineering~General programming languages}
\ccsdesc[300]{Social and professional topics~History of programming languages}
%% End of generated code


%% Keywords
%% comma separated list
\keywords{Agda, dependent types, specification}  %% \keywords is optional


%% \maketitle
%% Note: \maketitle command must come after title commands, author
%% commands, abstract environment, Computing Classification System
%% environment and commands, and keywords command.
\maketitle


\section{Introduction}

Agda 2 has been developed since 2005, and been released 2007.  So far,
no specification has been given.  This document attempts to specify
the core components of Agda.


\section{Syntax}
\label{sec:syntax}

\subsection{Terms}

We distinguish global names $f,F,D,c,R,\pi$ bound in the signature $\Sigma$ from
local variables $x,y,z$ bound in typing contexts $\Gamma$.  Local
variables are represented by de Bruijn indices in the implementation,
but for the sake of readability we stick to a named presentation here.
We silently rename bound variables to avoid clashes with free variables
(Barendregt convention).  We write $x \# E$ (``$x$ fresh for $E$'')
to express that $x$ is a
fresh variable with regard to syntactic entity $E$.

For orientation of the reader, we use different letters to represent
different purposes of global names.  However, they share the same name
space and are not distinguished syntactically.
\[
\begin{array}{l@{\hspace{2cm}}l}
  D & \text{data type name} \\
  R & \text{record type name} \\
  F & \text{data type or record type name} \\
  f & \text{function name} \\
\pi & \text{projection name  (overloaded)} \\
  c & \text{data/record constructor name (overloaded)} \\
\end{array}
\]
Projection and constructor names can be overloaded and are resolved by
the type checker.  We write $R.\pi$ for a resolved record projection
name, $R.c$ for a resolved record constructor name and $D.c$ for a
resolved data constructor name.

\emph{Terms} use a spine form for eliminations and thus are kept $\beta$ and
projection normal form.  This means that terms cannot be a
$\beta$-redex $(\lambda x.\,v)\,u$ nor a projection of a record value
$\cpi\,\vec v\,.\pi$.
\[
\begin{array}{lrl@{\hspace{2cm}}l}
t, u, v
  &::=& x\; \bar{e}      & \text{variables (eliminated by $\bar{e}$)} \\
  & | & f\; \bar{e}      & \text{defined function (eliminated by $\bar{e}$), includes postulates} \\
  & | & c\; \bar{v}      & \text{data constructor applied to arguments $\bar{v}$} \\
  & | & \cpi\; \bar{v}   & \text{record constructor with fields $\vec\pi$ applied to arguments $\bar{v}$} \\
  & | & \lambda x.\, v   & \text{lambda abstraction} \\
  & | & F\; \bar{v}      & \text{data/record type name applied to arguments $\bar v$} \\
  & | & \funT x U V      & \text{dependent function type} \\
  & | & s                & \text{Sort} \\
\end{array}
\]
We use uppercase letters $T,U,V$ for terms that stand for types.

\emph{Eliminations} $e$ exists for functions and records.
Functions are eliminated by application, records by projection.
\[
\begin{array}{lrl@{\hspace{2cm}}l}
e &::=& @u               & \text{application to term $u$} \\
  & | & .\pi             & \text{taking projection $\pi$} \\
\end{array}
\]
Other forms of elimination, like definiting a function by cases over
some data structure, need pattern matching, which is supported in
function declarations, but not in terms directly.

\emph{Sorts} $s$ are the types of types.
Here, we model a predicative hierarchy of universes $\Set_0 : \Set_1 : \Set_2 : \dots$.
\[
\begin{array}{lrl@{\hspace{2cm}}l}
s & ::= & \Set_n         &  \text{universe of types of level $n \in \NN$} \\
\end{array}
\]
We define \emph{sort supremum} \fbox{$s \sqcup s'$} by $\Set_i \sqcup \Set_j = \Set_{\max(i,j)}$.


\subsection{Telescopes and patterns}

\emph{Telescopes} $\Delta$ are like typing contexts $\Gamma$ (see below), but right-associative.
Telescopes are used \eg{} for parameter lists in declarations.
\[
\begin{array}{lrl@{\hspace{2cm}}l}
\Delta &::=& \cempty       & \text{empty telescope} \\
       &  |& (x : T)\Delta & \text{non-empty telescope; $\Delta$ can depend on $x$} \\
\end{array}
\]
Notation: iterated function types \fbox{$\Delta \to T$}, defined by recursion on $\Delta$:
\[
\begin{array}{rrl@{\hspace{2cm}}l}
\cempty \to T       &=& T &\\
(x : U)\Delta \to T &=& (x : U) \to (\Delta \to T) & \\
\end{array}
\]

\emph{Patterns} $p$ are used for definition by cases.
Patterns are made from variables $x$ and data constructors $c$,
but can also contain arbitrary terms $u$ that do not bind any variables.
The latter form is called \emph{inaccessible} pattern, or, due to the
concrete syntax used in Agda, \emph{dot} pattern.
\[
\begin{array}{lrl@{\hspace{2cm}}l}
p &::=& x              &   \text{variable pattern} \\
  & | & c\; \bar{p}    &   \text{constructor pattern} \\
  & | & \dotp u        &   \text{inaccessible pattern (aka dot pattern)} \\
\end{array}
\]

\emph{Copatterns} are to eliminations what patterns are to arguments.
We match eliminations against copatterns like we match arguments against patterns.
\[
\begin{array}{lrl@{\hspace{2cm}}l}
q &::= & @p   & \text{application pattern} \\
  &  | & .\pi & \text{projection  pattern}\\
\end{array}
\]

\subsection{Declarations}
\label{sec:declarations}

Agda has three main \emph{declaration} forms that introduce global names into
the signature $\Sigma$.  Function declarations introduce global
functions $f$ defined by pattern matching.  A function declaration
consists of a type signature $\vts$ and a list of function clauses
$\vec\vcl$.  The signature has to preceed the clauses, but between
them other declarations are allowed, in order to facilitate mutually
recursive definitions.  Data (type) declarations consist of a data
signature $\vds$ and a data definition $\vdd$, which have to appear in
this order but can also be appart, to realize inductive-recursive
definitions, for instance.  Similarly record (type) declarations
consist of a record signature $\vrs$ and a record declaration $\vrd$.
\[
\begin{array}{lrl@{\hspace{2cm}}l}
d  & ::=  & \mathit{ts} & \mbox{type signature}
\\ & \mid & \mathit{cl} & \mbox{function clause}
\\ & \mid & \mathit{ds} & \mbox{data signature}
\\ & \mid & \mathit{dd} & \mbox{data definition}
\\ & \mid & \mathit{rs} & \mbox{record signature}
\\ & \mid & \mathit{rd} & \mbox{record definition}.

\end{array}
 \]

\emph{Type signatures} consist of a global name $f$ for the function and its type $T$.
\[\begin{array}{lrll}
ts &::=& f : T &\\
\end{array} \]
A \emph{function clause} for $f$ consists of a copattern spine $\vec q$ and a right hand side $t$.
\[
\begin{array}{lrll}
cl &::=& f\; \bar{q} = t &\\
\end{array}
\]

A \emph{data signature} consists of a name $D$ for the data type,
a list of its parameters $\Delta$,
a list of itsindices $\Delta'$,
and its target sort $s$.
\[
\textbf{data}\; D\; \Delta : \Delta' \to s
\]
\emph{Data definitions} repeat the parameter telescope $\Delta$,
since parameters will be mentioned in the types $T$ of the constructors $c$.
\[
\textbf{data}\; D\; \Delta \;\textbf{where}\; \overline{c : T}
\]

\emph{Record signature}s carry, in contrast to data signatures,
no index telescope $\Delta'$, as Agda does not support indexed record types.
\[
\begin{array}{@{}l}
\textbf{record}\; R\; \Delta : s
\end{array}
\]
\emph{Record declarations} supply a record constructor name $c$
and a list of fields $\pi$ with their types $T$.
\[
\begin{array}{@{}l}
\textbf{record}\; R\; \Delta \; \textbf{where} \\
\qquad  \textbf{constructor}\; c \\
\qquad  \textbf{field}\; \overline{\pi : T} \\
\end{array}
\]

Declaration checking moves declarations to the signature $\Sigma$,
but not literally, but in a refined form.
A signature is a list of signature declarations (used internally).
\[
\begin{array}{lcl}
\Sigma & ::= &  \overline{d_\Sigma} \\
d_\Sigma & ::= & \datasig \\
& \mid & \recsig \\
& \mid & \funsig
\end{array}
\]
TODO: UPDATE SIGNATURE DECLARATIONS.




\section{Typing rules}
\label{sec:typing}


\subsection{Auxiliary judgements}

Hereditary substitution \fbox{$u[\sigma] = v$}
is defined simultaneously with active elimination (see below).
\begin{gather*}
\inferrule
  {\sigma(x) \bang \overline{e}[\sigma] = v}
  {(x \overline{e})[\sigma] = v}
\qquad
\inferrule
  {\overline{e}[\sigma] = \overline{e}'}
  {(f \overline{e})[\sigma] = f \overline{e}'}
\qquad
\inferrule
  {\overline{v}[\sigma] = \overline{v}'}
  {(h \overline{v})[\sigma] = h \overline{v}'}
  \ h ::= F \mid c \mid \cpi
\qquad
\rux{v[\sigma] = v'
  }{(\lam x v)[\sigma] = \lam x v'
  }{x \# \sigma}
\\
\rux{U[\sigma] = U' \qquad
     V[\sigma] = V'
   }{(\piT x U V)[\sigma] = \piT x {U'}{V'}
   }{x \# \sigma}
\qquad
\ru{}{s[\sigma] = s}
\end{gather*}
Active elimination \fbox{$t \bang \overline{e} = v$}\ .
\begin{gather*}
\ru
  { v[u/x] \bang \overline{e} = v'}
  {\lambda x.v \bang @u \overline{e} = v'}
\qquad
\inferrule
  {v_i \bang \overline{e} = v'}
  {\cpi \overline{v} \bang .\pi_i \overline{e} = v'}
\qquad
\ru{}{t \bang \cdot = t}
\qquad
\ru{}{x \overline{e} \bang \overline{e}' = x \overline{e}
  \overline{e}'}
\qquad
\ru{}{f \overline{e} \bang \overline{e}' = f \overline{e} \overline{e}'}
%% We let constructors, record and data types be fully applied.
% \qquad
% \ru{}{z \overline{v} \bang @ \overline{u} = z \overline{v}
%   \overline{u}}\ z ::= c \mid F
% \qquad
% \ru{}{D \overline{v} \bang @ \overline{u} = D \overline{v}
%   \overline{u}}
% \qquad
% \ru{}{R \overline{v} \bang @ \overline{u} = R \overline{v} \overline{u}}
\end{gather*}
Note that since constructor and data and record type terms are fully
applied, we do not need to give rules for adding further applications
to these.  Of course, function type and sort expressions cannot be applied either.
% \[
% \[
% \begin{array}{ll}
% &
% \\
% &
% \\
% \inferrule{c \overline{v} u \bang \overline{e} = v'}
% {\overline{v} \bang @u \overline{e} = v'}
% &
% \\
% &
% \text{ everything else undefined }
% \end{array}
% \]


Function type application \fbox{$T \twobang \vec u = U$}.
\[
  \ru{}{T \twobang . = T}
\qquad
  \ru{\subst u x V \twobang \vec u = T
    }{\funT x U V \twobang u \vec u = T}
\]

% Type elimination \fbox{$\Sigma \der (t : T) \twobang e = U$}.
% \[
%   \ru{}{\Sigma \der (t : \funT x U V) \twobang @u = \subst u x V}
% \qquad
%   \ru{\Sigma \der R.\pi : U
%     }{}
% \]

Signature lookup \fbox{$\Sigma \der z : T$}.
\begin{gather*}
  \ru{\funsig \in \Sigma
    }{\Sigma \der f : T}
\\
  \ru{\datasig \in \Sigma
    }{\Sigma \der D : \Delta \to \Delta' \to s}
\qquad
  \ru{\datasig \in \Sigma
    }{\Sigma \der D.c_i : \Delta \to T_i}
\\
  \ru{\recsig \in \Sigma
    }{\Sigma \der R : \Delta \to s}
\\
  \ru{\recsig \in \Sigma
    }{\Sigma \der R.c : \Delta \to T}
\qquad
  \ru{\recsig \in \Sigma
    }{\Sigma \der R.\pi_i : \Delta \to T_i}
\end{gather*}




\subsection{Typing judgements for expressions}


Typing contexts.
\[
\begin{array}{lcl}
\Gamma & ::= & \cdot \mid \Gamma, x:T \\
\Delta & ::= & \cdot \mid x:T, \Delta
\end{array}
\]

Well-typed contexts \fbox{$\ders \Gamma$}.
\[
% Contexts
\ru{}{\ders \cdot}
\qquad
\inferrule
  {\ders \Gamma \qquad \Gamma \ders T:s}
  {\ders \Gamma, x : T} \ x \not\in \dom(\Gamma)
\]

Well-typed telescopes \fbox{$\Gamma \ders \Delta$}.
\[
% Telescopes
\inferrule
  {\ders \Gamma}
  {\Gamma \ders \cdot}
\qquad
\inferrule{\Gamma \ders T : s \qquad \Gamma, x:T \ders \Delta}
{\Gamma \ders (x:T) \Delta}
% \qquad x \in \dom{\Delta} \\
\]

Spine typing \fbox{$\Gamma \mid u : U \ders \overline{e} : T$}
($u$ is neutral).
\begin{gather*}
% Check spine
\ru{}
{\Gamma \mid u:U \ders \cdot : U}
\\
\inferrule
  {\Gamma \ders u:U \qquad \Gamma \mid t@u:T[u/x] \ders \overline{e} : V}
  {\Gamma \mid t : \Pi x:U.T \ders @u \overline{e}:V}
\qquad
\ru{\Sigma \der R.\pi = T \qquad
    T \twobang (\vec u,t) = U \qquad
    \Gamma \mid t.\pi : U \ders_\Sigma \overline{e} : V
  }{\Gamma \mid t :  R\overline{u}  \ders .\pi \overline{e} : V}
% \inferrule{ \Gamma \vdash (t:R\overline{u}).\pi = T'
% \qquad \Gamma \mid t.\pi :T' \ders \overline{e}:V}
% { \Gamma \mid t : R \overline{u}  \ders .\pi \overline{e}:V}
% \\
% \Sigma \vdash (t:R\overline{u}).\pi = T
% \\
% (t: \Sigma x:A.B).\proj_2 = B[t.\proj_1/x] &
% \\
% \inferrule{T[u/x] \twobang \overline{e} = v}
% {(\Pi x:U.T) \twobang @ u \overline{e} = v}
\\
\ru{\Gamma \mid u : U \ders \vec e : V \qquad
    \Gamma \ders U = U' : s
  }{\Gamma \mid u : U' \ders \vec e : V}
\end{gather*}


Term typing \fbox{$\Gamma \ders t : T$}
\begin{gather*}
\inferrule
  {(x:U)\in \Gamma \qquad \Gamma \mid x:U \ders \overline{e}:T}
  {\Gamma \ders x \overline{e} : T}
\qquad
\ru{\Sigma \vdash f : T \qquad
    \Gamma \mid f : T \ders \vec e : V
   }{\Gamma \ders f \vec e : V}
\\
\ru{\Sigma \der D.c : T \qquad
    T \twobang \vec u = T' \qquad
    \Gamma \mid c : T' \ders \vec v : D \, \vec u \, \vec t
  }{\Gamma \ders c\,\vec v : D\,\vec u\,\vec t}
\qquad
\ru{\Sigma \vdash z : T \qquad
    \Gamma \mid z : T \ders \vec u : V
   }{\Gamma \ders z \, \vec u : V} \ z ::= D \mid R
\\
\ru{\Gamma, x \of U \ders v : V
  }{\Gamma \ders \lambda x.v : \funT x U V}
\qquad
\ru{\Gamma \ders U : s \qquad
    \Gamma, x \of U \ders V : s'
  }{\Gamma \ders \piT x U V : s \sqcup s'}
\end{gather*}


Conversion
\fbox{$\Gamma \ders t = t' : T$}\ .
\begin{gather*}
%
\ru{f\,\overline q = t \in \Sigma \qquad
    (\overline{\embp q})[\sigma] = \overline e \qquad
    t[\sigma] = v \qquad
    \Gamma \ders f\,\overline e : T \qquad
    \Gamma \ders v : T
  }{\Gamma \ders f \overline e = v : T}
% \ru{\Sigma \der t \red t' \qquad
%     \Gamma \ders t : T \qquad
%     \Gamma \ders t' : T
%   }{\Gamma \ders t = t' : T}
\\
\ru{\Gamma \ders t : \funT x U V
  }{\Gamma \ders t = \lambda x. t\, x : \funT x U V}
\qquad
\ru{\Gamma \ders t : R\, \vec u \qquad
    \Sigma \der R.\cpi : T
  }{\Gamma \ders t = \cpi\,\overline{t .\pi} : R\,\vec u}
\\
\end{gather*}
And many boring rules (equivalence, congruence)
and rules for elimination equality
\fbox{$\Gamma \mid t : T \ders \vec e = \vec e' : V$}


\subsection{Declaration typing}

(Co)pattern variables \fbox{$PV(p) = \bar{x}$}
\begin{gather*}
PV(x) = \{ x \} \qquad PV (\lceil v \rceil) = \emptyset \qquad PV(c\; \bar{p}) = PV(\bar{p}) \\
PV(\bar{p}) = \biguplus_i PV(p_i) \\
PV(.\pi) = \emptyset \qquad PV(@p) = PV(p) \\
PV(\bar{q}) = \biguplus_i PV(q_i) \\
\end{gather*}


Embedding of (co)patterns into terms \fbox{$\lceil p \rceil = t$}
\begin{gather*}
\lceil x \rceil = x \qquad
\lceil \lfloor v \rfloor \rceil = v \qquad
\lceil c\; \bar{p} \rceil = c\; \lceil \bar{p} \rceil \\
\lceil .\pi \rceil = .\pi \qquad
\lceil @p \rceil = @ \lceil p \rceil
\end{gather*}


%Pattern typing \fbox{$\Gamma \ders p : U \leadsto \Gamma'$}
%\begin{gather*}
%\ru{\Gamma \ders U : s}
%   {\Gamma \ders x : U \leadsto (x : U)} x \not\in dom(\Gamma) \\
%\ru{\Sigma \der D.c : T \qquad T\twobang\bar{u} = T' \qquad \Gamma \mid c : T \ders @\bar{p} : D\; \bar{u}\; \bar{v} \leadsto \Gamma'}
%   {\Gamma  \ders c\; \bar{p} : D\; \bar{u}\; \bar{v} \leadsto \Gamma'}
%\end{gather*}
%
%
%Copattern typing \fbox{$\Gamma \mid f : T \ders \bar{q} : U \leadsto \Gamma'$}
%\begin{gather*}
%\ru{\Gamma \vdash p : U \leadsto \Gamma_1 \qquad \Gamma,\Gamma' | h\; @p : V[p/x] \ders \bar{q} : T  \leadsto \Gamma_2}
%   {\Gamma \mid h : \piT x U V \ders @p\; \bar{q} : T \leadsto \Gamma_1 , \Gamma_2}
%\end{gather*}

Clause typing \fbox{$f : T \ders cl$}
\begin{gather*}
\ru{PV(\bar{q}) = dom(\Gamma) \qquad
    \Gamma | f : T \ders \lceil \bar{q} \rceil : U \qquad
    \Gamma \ders t : U}
   {f : T \ders f\; \bar{q} = t}
\end{gather*}

Constructor typing \fbox{$\Delta | U : \Delta' \to s \ders c : T$}
\begin{gather*}
\ru{\Delta \ders \Gamma \to T : s \qquad
    \Delta,\Gamma \ders \bar{v} : \Delta' \qquad
    \Delta,\Gamma \ders T = U\; \bar{v} : s}
   {\Delta | U : \Delta' \to s \ders c : \Gamma \to T}
\end{gather*}

% TODO: add syntax for lone declarations without a definition (i.e. with no 'where').
Declaration typing \fbox{$\Sigma \der d \leadsto \Sigma'$}
\begin{gather*}
\ru{f \not\in \Sigma \qquad \cdot \ders T : s}
   {\Sigma \der f : T \leadsto \Sigma, function\; f : T\; where\; \cdot}
\\
\ru{\forall i.\; f : T \ders cl_i}
   {\Sigma[function\; f : T\; where\; \cdot] \der \overline{cl} \leadsto \Sigma[function\; f : T\; where\; \overline{cl}]}
\\
\ru{D \not\in \Sigma \qquad \ders \Delta \qquad \Delta \ders \Delta'}
   {\Sigma \der data\; D\; \Delta : \Delta' \to s \leadsto \Sigma, data\; D\; \Delta : \Delta' \to s}
\\
\ru{\forall i.\; \Delta | D\; \Delta : \Delta' \to s \vdash c_i : T_i}
   {\Sigma[data\; D; \Delta : \Delta' \to s] \ders data\; D\; \Delta\; where\; \overline{c : T} \leadsto \Sigma[data\; D\; \Delta : \Delta' \to s\; where\; \overline{c : T}]}
\\
\ru{R \not\in \Sigma \qquad \ders \Delta}
   {\Sigma \ders record\; R \; \Delta : s \leadsto \Sigma, record\; R\; \Delta : s}
\\
\ru{\Delta \ders \overline{(\hat{\pi} : T)} \to R\; \Delta : s \qquad
    U_i = \Delta \to (x : R\; \Delta) \to T_i[x.\pi_j / \hat{\pi_j}] }
   {\Sigma[record\; R\; \Delta : s] \ders record\; R\; \Delta\; where\; constructor\; c, \overline{field\; \hat{\pi} : T} \\  \leadsto \Sigma[record\; R\; \Delta : s\; where\; c : \Delta \to \overline{(\pi : T)} \to R\; \Delta; projections\; \overline{\pi : U}]} x \not\in dom(\Delta)
\end{gather*}




\section{Evaluation}
\label{sec:evaluation}


Matching \fbox{$\Sigma \vdash e/q = \sigma$}\ .
\begin{gather*}
\ru{}{\Sigma \vdash v/\lfloor u \rfloor = \cdot} % dot pattern
\qquad
\ru{}{\Sigma \vdash v/x = [v/x]} % var pat
\qquad
\ru
  {\Sigma \vdash \overline{v}/\overline{\sigma} = \sigma}
  {\Sigma \vdash c\overline{v}/c\overline{p} = \sigma}  % con pat
\qquad
\ru{v \bang \pi_i / p_i = \sigma}
{\Sigma \vdash v / \cpi \overline{p} = \sigma} % record pat
\qquad
\ru{}{\Sigma \vdash .\pi/.\pi = \cdot} % proj pat
\\
\ru{\Sigma \vdash v \to v' \qquad \Sigma \vdash v'/c\overline{p} = \sigma}
{\Sigma \vdash v / c \overline{p} = \sigma} % reduce
\end{gather*}
% \[
% \begin{array}{ll}
% \overline{e}/\overline{q} = \sigma  &
% \Sigma \vdash e/q = \sigma
% \end{array}
% \]

% \[
% \begin{array}{ll}
% \inferrule{}
% {\Sigma \vdash v/\lfloor u \rfloor = \cdot} % dot pattern
% &
% \inferrule{}
% {\Sigma \vdash v/x = [v/x]} % var pat
% \\
% \inferrule{\Sigma \vdash \overline{v}/\overline{\sigma} = \sigma}
% {\Sigma \vdash c\overline{v}/c\overline{p} = \sigma}  % con pat
% &
% \inferrule{}
% {\Sigma \vdash .\pi/.\pi = \cdot} % proj pat
% \\
% \inferrule{\Sigma \vdash v \to v' \qquad \Sigma \vdash v'/c\overline{p} = \sigma}
% {\Sigma \vdash v / c \overline{p} = \sigma} % reduce
% &
% \inferrule{v \bang \pi_i / p_i = \sigma}
% {\Sigma \vdash v / c_{\overline \pi} \overline{p} = \sigma} % record pat
% \end{array}
% \]

Weak head reduction \fbox{$\Sigma \vdash t \to t'$}
\[
\begin{array}{lll}
\inferrule{f \overline q = t \in \Sigma \qquad \Sigma \vdash \overline{e}/\overline{q} = \sigma}
{\Sigma \vdash f \overline{e} \overline{e}' \to t[\sigma] \bang \overline{e}'}
&
\ru{}
{\Sigma \vdash \cdot/\cdot = \cdot} %empty
&
\inferrule{\Sigma \vdash e/q = \sigma \qquad \Sigma \vdash \overline{e}/\overline{q} = \sigma'}
{\Sigma \vdash e, \overline{e}/q, \overline{q} = \sigma \uplus \sigma'}%patterns
\end{array}
\]



\section{Coverage}
\label{sec:coverage}

\section{Termination}
\label{sec:termination}

\section{Positivity}
\label{sec:positivity}

\section{Extensions}
\label{sec:extensions}

\subsection{Extended record declarations}

Record types
\begin{verbatim}
record R Γ : s where
  [constructor c]
  [(co)inductive]
  [(no-)eta-equality]
  rd*
\end{verbatim}

Record declaration
\begin{verbatim}
rd ::= field π : T
     | ts
     | cl
\end{verbatim}


%% Acknowledgments
\begin{acks}                            %% acks environment is optional
                                        %% contents suppressed with 'anonymous'
  %% Commands \grantsponsor{<sponsorID>}{<name>}{<url>} and
  %% \grantnum[<url>]{<sponsorID>}{<number>} should be used to
  %% acknowledge financial support and will be used by metadata
  % extraction tools.
  % This material is based upon work supported by the
  % \grantsponsor{GS100000001}{National Science
  %   Foundation}{http://dx.doi.org/10.13039/100000001} under Grant
  % No.~\grantnum{GS100000001}{nnnnnnn} and Grant
  % No.~\grantnum{GS100000001}{mmmmmmm}.  Any opinions, findings, and
  % conclusions or recommendations expressed in this material are those
  % of the author and do not necessarily reflect the views of the
  % National Science Foundation.
\end{acks}


%% Bibliography
%\bibliography{bibfile}


%% Appendix
% \appendix
% \section{Appendix}


\end{document}
